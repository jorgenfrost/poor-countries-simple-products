\section{Cleaning ASI data}
The ASI is distributed by the Ministry of Statistics and Programme Implementation, Government of India, (MOSPI) as ten blocks for every year. These blocks require substantial cleaning and harmonization of variables. I first create a base sample of plant-year observations. From this base sample I again filter observations depending on the analysis. I first outline how I get my base sample, then the different analysis sample.

I first create the base sample. Plants can be included in the survey even if they are reported to be closed or are missing. I drop all observations not listed as open. I also drop all factories that are not listed as in a manufacturing sector and observations that don't report revenues (defined as the total gross sale value of all production output). Finally, I drop observations if they are exact copies of other observations.

After the initial filtering process, there can still be observations that have misreported values of specific variables. When analysing these variables, I further limit the sample using a "flagging"-system\footnote{This method was inspired by \cite{allcott_how_2016}.}. 

I assign observations a "revenue share" flag if a plant's labour or intermediate input costs are more than two times their revenues and if electricity purchases is more than revenues. In addition, I assign "change" flags based on the difference between observations of the same plant. If the amount reported by a plant changes by more than 3.5 log values from the observation before and the observation after, they are assigned a flag. "Change" flags are computed for revenues, number of employees and use of electricity (kWh). For example: if a plant has a (real price) ln(revenue) of 15 in 2012, of 7 in 2013, and of 11 in the 2014, the 2013 value gets flagged, but if the 2014 value was 9 instead, it would not. For observations on either end (first or last observation of a plant) I assign change flags based only on the change from the previous or next observation. Each time I run an analysis involving any of the flagged variables (both revenue-share and change flags) as an outcome, I exclude observations that are flagged. I drop all the observations that have two or more flags completely.

\subsection{Product concordance for ASI}%
\label{sub:product_concordance_asi}

As mentioned in the data-section, the ASI lists products according to two different classification methods. In earlier years (before 2010-2011) the ASICC classification is used, whereas later years lists product by their NPCMS-2011 code. The standard nomenclature for international trade, however, is the Harmonized System classification (HS). Since I assign complexity to plants by their the products they produce, and since I calculate the complexity of products by their position in the international trade network, I need to map the HS system to the codes used in the ASI.

This is rather round-about process. The reason behind the shift from ASICC to NPCMS-2011 is that the early scheme was severely flawed in the grouping-classification and was poorly suited to international comparison. This means that the mapping between ASICC and NPCMS-2011 is imperfect. The NPCMS-2011 mapping is based directly on the international standard Central Product Classification, which again is different from the Harmonized System used in trade-accounting. I first match all products from the ASICC years to the NPCMS-2011 classification with the concordance table provided by MOSPI \footnote{http://www.csoisw.gov.in/CMS/En/1027-npcms-national-product-classification-for-manufacturing-sector.aspx}. I then turn the NPCMS-2011 codes into the CPC-2 classification by removing the last two digits (which are India specific). I use the concordance table supplied by UNSD to map the CPC-2 codes to HS-2007. Finally, I use turn the HS-2007 codes into HS-1996 to match the trade data. 

Often, one product code from the source classification maps to two different codes in the destination classifications. There is no way to solve this issue completely. Instead, I create two mappings: a "strict" and a "lenient" match. The "strict" match uses only products that have a non-partial match and leaves other products as missing. The "lenient" approach assigns the first of the partial mappings as a match. Since the difference is typically very small between partially mapped products, is is usually feasible to purposely "mis-assign" the products to a mapping that exists, rather than drop it altogether. For instance, the ASICC listings of "Lobsters, processed/frozen" (11329), "Prawns, processed/frozen" (11331), "Shrimps, processed/frozen" (11332) all map to two different NPCMS-2011 codes: "Crustaceans, frozen" (212500) and "Crustaceans, otherwise prepared" (212700). Similarly, "Butter" (11411) maps to three different kinds of butter (based on cattle-milk, buffalo-milk, or other milk) in the NPCMS-2011 system. While not particularly rigorous, very little information should be lost on the complexity of the production output between these three mappings. Indeed, many such categories will be clubbed together anyhow when converting NPCMS-2011 to Harmonized System codes. It is worth noting that I use the "strict"/"lenient" approach throughout the concordance chain. This means a substantial product loss in the "strict" approach: products from the ASICC classification (five digits) that might be together in the final Harmonized System code can be dropped because they map to two different NPCMS-2011 codes (that are seven digits vs the four I use in the HS-code). At any rate, while the observations are substantially reduced in some states, the distribution of plant complexity changes very little (see figure TODO). I therefore use the lenient approach in my main analysis.


%% DENSITY USING DIFFERENT METRICS (WEIGHTED)
\begin{figure}[htpb]
	\centering
	\includegraphics[width=1\linewidth]{figures/appendix/density_metrics_combined.pdf}
	\caption{}%
	\label{fig:}
\end{figure}

%% DENISTY USING DIFFERENT METRICS ACROSS DIFFERENT STATES (WEIGHTED)
\begin{figure}
     \centering
     \begin{subfigure}[b]{0.45\textwidth}
         \centering
         \includegraphics[width=\textwidth]{figures/appendix/density_by_state_avg_rpca.pdf}
	 \caption{}
         \label{fig:}
     \end{subfigure}
     \hfill
     \begin{subfigure}[b]{0.45\textwidth}
         \centering
         \includegraphics[width=\textwidth]{figures/appendix/density_by_state_max_rpca.pdf}
	 \caption{}
         \label{fig:interaction_sample_max}
     \end{subfigure}
     \caption[]{}
        \label{fig:}
\end{figure}

\begin{figure}
     \centering
     \begin{subfigure}[b]{0.45\textwidth}
         \centering
         \includegraphics[width=\textwidth]{figures/appendix/density_by_state_avg_rca.pdf}
	 \caption{}
         \label{fig:}
     \end{subfigure}
     \hfill
     \begin{subfigure}[b]{0.45\textwidth}
         \centering
         \includegraphics[width=\textwidth]{figures/appendix/density_by_state_max_rca.pdf}
	 \caption{}
         \label{fig:interaction_sample_max}
     \end{subfigure}
     \caption[]{}
        \label{fig:}
\end{figure}

%% HISTOGRAM OF COMPLEXITY (UNWEIGHTED) USING DIFFERENT METRICS IN DIFFERENT STATES
\begin{figure}
     \centering
     \begin{subfigure}[b]{0.45\textwidth}
         \centering
         \includegraphics[width=\textwidth]{figures/appendix/histogram_avg_rca.pdf}
	 \caption{}
         \label{fig:}
     \end{subfigure}
     \hfill
     \begin{subfigure}[b]{0.45\textwidth}
         \centering
         \includegraphics[width=\textwidth]{figures/appendix/histogram_max_rca.pdf}
	 \caption{}
         \label{fig:interaction_sample_max}
     \end{subfigure}
     \caption[]{}
        \label{fig:}
\end{figure}

\begin{figure}
     \centering
     \begin{subfigure}[b]{0.45\textwidth}
         \centering
         \includegraphics[width=\textwidth]{figures/appendix/histogram_avg_rpca.pdf}
	 \caption{}
         \label{fig:}
     \end{subfigure}
     \hfill
     \begin{subfigure}[b]{0.45\textwidth}
         \centering
         \includegraphics[width=\textwidth]{figures/appendix/histogram_max_rpca.pdf}
	 \caption{}
         \label{fig:interaction_sample_max}
     \end{subfigure}
     \caption[]{}
        \label{fig:}
\end{figure}

%% SCATTERPLOT COMPLEXITY OF A PLANT USING DIFFERENT METRICS (WEIGHTED)
\begin{figure}[htpb]
	\centering
	\includegraphics[width=1\linewidth]{figures/appendix/scatterplot_plant_complexity_across_metrics.pdf}
	\caption{}%
	\label{fig:}
\end{figure}

\section{Shortage data}

%% DISTRIBUTION OF SHORTAGES
\begin{figure}[htpb]
	\centering
	\includegraphics[width=1\linewidth]{figures/appendix/shortage_distribution.pdf}
	\caption{}%
	\label{fig:}
\end{figure}
